% !TEX root = master.tex

This study uses \iband imagery from both the Tenth Data Release of the Sloan Digital Sky Survey (SDSS DR10) and the latest MegaCam release to study truncated lenticular (S0) galaxies in the Coma cluster. Using an image from each camera we take two 1D profiles along the major axis of each. We develop algorithms for measuring sky, finding the break point and characterising the truncation for use with the 1D profiles. Using a sample of known disk galaxies, we refine the sample to include only 66 S0s with disk dominated outer disks and axis ratio $b/a > 0.4$. We find that $74^{+5}_{-5}\%$ of S0s are anti-truncated and $24^{+6}_{-5}\%$ exhibit a pure exponential. Comparing to previous work in the Virgo cluster. We agree that there is a negligible fraction of truncated/down-bended disks in the cluster environment. But we also measure no variation of type-I disks between field and Coma cluster environments. We observe a weak trend between truncated disk frequency and distance from cluster centre: $80^{+9}_{-7}\%$ of disks at $R_{clust} < 68$ Mpc are anti-truncated as opposed to $60^{+14}_{-12}\%$ at $R_{clust} > 68$ Mpc.
