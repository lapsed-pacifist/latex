% !TEX root = master.tex

We have used SDSS and MegaCam \iband images to investigate truncations in S0s present in the Coma Cluster. Using a refined sample of 66 galaxies exhibiting a wide range of bulge and classical disk parameters, we have measured taken two 1D profiles along the the major axis for each image. The resulting four profiles were first fit with a \sersic bulge + exponential disk model with the requirement that disk dominates at large $R$. Then the truncation point was found and a truncated disk fit yielding a distribution of truncated, anti-truncated and classical exponential disks. The four profiles are then combined with their bootstrapped confidence intervals to give overall parameters for the galaxy.

Our main conclusions are:
\begin{itemize}
	\item 1D decomposition with truncation can be easily automated and requires little computational effort. The iterative sky finding algorithm is also effective at measuring mean sky levels at large $R$.
	\item There is a tendency for this procedure to place break radii towards the end of the profile i.e. where the sky starts to dominate. This needs to be corrected.
	\item We find strong evidence for the suppression of formation of type-II truncated galaxies in the cluster environment, in agreement with values found in the Virgo cluster \citep{erwin_strong_2012}
	\item We find that type-III anti-truncated disks are significantly more frequent than type-I single exponential disks in the cluster. 
	\item We observe brighter and more centrally compact bulges in type-IIIs than type-Is and conclude that this supports the galaxy harassment hypothesis of type-III formation
	\item We measure a weak trend of increasing type-III fraction with decreasing cluster-centric radius suggesting that type-IIIs form more frequently in high density environments
\end{itemize}

Although it is possible to characterise truncations with this method, there is significant error combining four profiles. A 2D fit and decomposition would undoubtedly render more robust and accurate results.