% !TEX root = master.tex

% talk about initial disk fitting- disk dominated section, fit without weights etc

% Fitting a galaxy with no truncations as above, is helpful to quantify the bulge size. It also provides a reliable, if not robust, measure of truncation type through its residuals. - tighter weighting results in a good measure for truncation type further out

\subsection{Sky Detection} % (fold)
\label{sub:sky_detection}
The automated detection routine as describe above successfully finds the sky dominated area with all sample light profiles. The sky value obtained from our detection algorithm differed from the values delivered from the SDSS/MegaCam pipeline by as much as 5\%, though typically only differs by about 0.3\%.  Sky uncertainty as the standard deviation differs from the pipeline uncertainties ($\sqrt{sky}$) by around 40\%. 
\begin{figure}[h]
	\centering
	\makebox[0.7\columnwidth]{\includegraphics[width=0.7\textwidth]{figs/sky_hist.png}}
	\caption{\footnotesize{Histogram of the differences between our sky values and MegaCam/SDSS}}
\end{figure}

% subsection sky_detection (end)


\subsection{Example Decomposition} % (fold)
\label{sub:example_decompositions}
We present a typical galaxy decomposition and classifications for a disk exhibiting upbending to illustrate the classification process. ID 1237667444048265310 is a typical example of a type-III classified galaxy.

\begin{sidewaysfigure}[p]
	\centering
	\makebox[1.\columnwidth]{\includegraphics[width=1.5\textwidth]{figs/example_decomp.png}}
	\label{fig: The mega profiles of 1237667444048265310}
	\caption{\footnotesize{The surface brightness profile for the MegaCam image of 1237667444048265310. The plot shows the inner and outer disks in light and dark dashed lines respectively. The bulge is illustrated in green and the original pre-truncated disk is shown as a red dotted line. The total is depicted with a thin black line.
		 Unreal (negative counts) values are shown with red arrows and the critical sky value as the red region. The dashed lines at the top and bottom denote the lower sky limit, lower break limit, upper break limit and bulge cut-off radius, working inwards. Normalised magnitude residuals are shown at the bottom.}}
\end{sidewaysfigure}
Figure \ref{fig: The mega profiles of 1237667444048265310} shows the surface brightness profile of 1237667444048265310. Both MegaCam profiles, representing the different major axes, are completely independently fit. This galaxy is a typical example of an error in truncation detection. Usually there is one profile per galaxy which exhibits this undesirable behaviour. This problem cannot be remedied by restricting the parameters in the fit, since any more restriction tends to yield unreal solutions such as lack of bulge.

Firstly, a simple bulge + disc is fit to the profile, resulting in a typical profile with a disc scale length of $\sim 6$ arcsec and bulge effective surface brightness of $\sim 20$ mag. 

On the other hand, the bulge shape -characterised by the \sersic index- varies between SDSS and MegaCam fits. Though this changes the shape of the bulge entirely, its surface brightness remains small enough ($\mu_B < \mu_D + 0.2$ mag) as to not have an impact the total at large $R$.

Next, the automated truncation detection yields a potential break point and the two disks are parameterised. The routine classifies these individual fits as Type-III (upbended) since the lower limit on the outer scale length does not fall below the upper limit on the inner scale length. 

As seen in the residual plot accompanying the fits, there is poor fit at the centre of the galaxy, where the bulge is strongest. Since the truncation fit only subtracts the bulge's contribution to the total magnitude, it is likely that the poor central fit is due to neglection of bulge light.

The next step is to combine the images to yield overall parameters for the entire galaxy. However, although there is a good agreement of pre-truncated-fit -called `classic' from now on- disk parameters between cameras there is a discrepancy in the \sersic index of the bulge. SDSS fits a small $n\approx 0.5$ bulge whilst MegaCam finds an expansive $n\approx 1.5$ bulge. 

This discrepancy error is a major problem when combining results as averaging the parameters does not yield an accurate `one-size-fits-all' solution. The truncation is commonly classified differently between cameras. We proceed by combining their bootstraps instead. This does not solve the problem of not yielding an accurate solution but it does provide an accurate error on the result, unlike a weighted arithmetic mean \citep{andrae_error_2010}. 

Defining the truncation strength as 
\begin{equation}
	S_h =  \frac{h_{outer} - h_{inner}}{h_{inner}},
\end{equation}
we find that $S_h = 1.3^{+0.9}_{-0.7} > 0$ for the combined profile and so is classified as Type-III.

% subsection example_decompositions (end)

\subsection{Sample Properties}
\begin{sidewaysfigure}[ph]
	\makebox[1.\columnwidth]{\includegraphics[width=1.\textwidth]{figs/props_hist.png}}
	\caption{\footnotesize{Proportion of both truncated and untruncated disks as a function of various parameters where available}}
	\label{fig: prop_hist}
\end{sidewaysfigure}
Though all profiles have been successfully fit with a disk dominated outer profile, it was not always real. There were frequent situations where a disc with perturbation (Allen type-6) were fit. These galaxies usually had what appears to be previously unseen spiral patterns, wide bars or simply a bulge dominated outer profile. Since these components are not part of this investigation, these galaxies were discarded.

We restricted ourselves to using galaxies with inclination (axis ratio) $b/a > 0.4$ and bulge size $R_{e,bulge} < 0.2R_{e,disk}$. This removed a great many bulge dominated systems, shrinking our sample to 83 galaxies. The initial sample of profiles was also filtered by removing galaxies whose profiles yield hugely disagreeing results. This was done by eye and the sample was refined to 66 reasonable fits. 

Our sample exhibits a range of bulge sizes, showing that our restrictions on bulge parameters relative to the disk has had little impact on the diversity of our sample. With a typical bulge of $R_e = 1.0\pm 0.9$ arcsec and $n = 0.8\pm 0.7$, the bulge clearly tends to be less luminous that the disk in truncated systems and more centrally concentrated than in type-Is ($R_e = 2\pm 1$ arsec and $n = 1.5 \pm 0.9$). The type-III preference for a more centralised bulge suggests that the upbending is not due to bulge influence at large $R$, which earlier, may have escaped notice. 

In Type-III disks, the central surface brightness of the outer disk must be greater than that of the inner disk. This can clearly be seen in the sharp divide between the distributions in figure \ref{fig: prop_hist}. The typical inner central surface brightness is $20.0\pm 0.8$ mag whereas in the outer disk it spans a wider range of values with a higher mean $24 \pm 2$ mag.

In truncated galaxies, the break point takes a wide range of radii and the strength of the break exhibits a peak at 2. The small number of truncated galaxies with low break strength can be attributed to the classification process, since a low break strength is more likely to lie within $\sim 1\sigma$ of the classical disk and hence classified as untruncated. 

\begin{figure}[h]
	\centering
	\includegraphics[scale=0.5]{figs/fraction_vs_cluster_radius}
	\caption{\footnotesize{Fraction of upbended S0s binned with equal distance bins. Error bars show 68\% confidence intervals \citep{wilson_probable_1927}}}
	\label{fraction vs dist}
\end{figure}

We find that  $74^{+5}_{-5}\%$ of S0s in the Coma cluster are anti-truncated type-IIIs and a corresponding $24^{+6}_{-5}\%$ are untruncated. The remaining 2\% of type-IIs consist of one galaxy (see Discussion for further information). 

% subsection Sample Properties (end)

\subsection{Correlations} % (fold)
\label{sub:correlations}
We found no statistically significant evidence of a trend between distance from cluster centre and bulge and classical disk parameters . However, there is a parabolic region of $30 \lesssim R_{clust} \lesssim 60$ arcsecs, $28 \lesssim \mu_{0,outer} \lesssim 25$ mag in which there are no truncated galaxies.

There is a slight trend for truncated galaxies to be found at lower cluster radii seen in figure \ref{fig: prop_hist} but this is not statistically significant.

We found evidence of a weak trend between the bulge magnitude and the break radius. A Pearson correlation coefficient of 0.26 indicates an increase in bulge magnitude as the break appears further out in $R$.
\begin{figure}[ht]
	\centering
	\label{bulge/break correlation}
	\makebox[0.8\columnwidth]{\includegraphics[width=0.8\textwidth]{figs/bulge_mag_vs_brk.png}}
	\caption{\footnotesize{A correlation plot between effective bulge magnitude $\mu_e$ and break radius $R_{brk}$ with truncated galaxies shown in red and untruncated galaxies shown in blue for comparison.}}
\end{figure}
Type-I S0s exhibit a tighter correlation between their potential break radius and bulge magnitude. This demonstrates that this trend is likely an artefact of the fitting process.

A weak trend in fraction of type-III against cluster-centric radius is observed when binning the galaxies into inner, outer and intermediate radii (see figure \ref{fraction vs dist}). The fraction of truncated S0s at the outskirts ($R > 60 $Mpc) is much lower than that of the innermost region.

% subsection correlations (end)
